% LaTeX Curriculum Vitae Template
%
% Copyright (C) 2004-2009 Jason Blevins <jrblevin@sdf.lonestar.org>
% http://jblevins.org/projects/cv-template/
%
% You may use use this document as a template to create your own CV
% and you may redistribute the source code freely. No attribution is
% required in any resulting documents. I do ask that you please leave
% this notice and the above URL in the source code if you choose to
% redistribute this file.

\documentclass[letterpaper]{article}

\usepackage{hyperref}
\usepackage{geometry}
\usepackage{tabularx}

% Comment the following lines to use the default Computer Modern font
% instead of the Palatino font provided by the mathpazo package.
% Remove the 'osf' bit if you don't like the old style figures.
\usepackage[T1]{fontenc}
\usepackage[sc,osf]{mathpazo}

% Set your name here
\def\name{Julian Mak}

% The following metadata will show up in the PDF properties
\hypersetup{
  colorlinks = true,
  urlcolor = black,
  pdfauthor = {\name},
  pdfkeywords = {fluids, oceanography, mathematics},
  pdftitle = {\name: Curriculum Vitae},
  pdfsubject = {Curriculum Vitae},
  pdfpagemode = UseNone
}

\geometry{
  body={6.5in, 8.5in},
  left=1.0in,
  top=1.25in
}

% Customize page headers
\pagestyle{myheadings}
\markright{\name -- CV}
\thispagestyle{empty}

% Custom section fonts
\usepackage{sectsty}
\sectionfont{\rmfamily\mdseries\Large}
\subsectionfont{\rmfamily\mdseries\itshape\large}

% Other possible font commands include:
% \ttfamily for teletype,
% \sffamily for sans serif,
% \bfseries for bold,
% \scshape for small caps,
% \normalsize, \large, \Large, \LARGE sizes.

% Don't indent paragraphs.
\setlength\parindent{0em}

% Make lists without bullets
\renewenvironment{itemize}{
  \begin{list}{}{
    \setlength{\leftmargin}{1.5em}
  }
}{
  \end{list}
}

\begin{document}

% Place name at left
{\huge \name}

% Alternatively, print name centered and bold:
%\centerline{\huge \bf \name}

\vspace{0.25in}
%=================================================================
\begin{tabular}{lll}
  Associate Professor & Email: & {\tt jclmak@ust.hk} \\
  Department of Ocean Sciences & Homepage: & \href{https://julianmak.github.io/index.html}{\tt
https://julianmak.github.io/index.html} \\
  Hong Kong University of Science and Technology & Tel: & +852 3469 2823
\end{tabular}
%=================================================================
\section*{Research interests}
\begin{itemize}
\item Geophysical and astrophysical fluid dynamics
\begin{itemize}
	\item[--] Wave/eddy-mean flow interaction
	\item[--] Baroclinic dynamics/turbulence
	\item[--] Development of numerical ocean models
	\item[--] Applied and computational mathematics (particularly optimisation problems)
	\item[--] Magnetohydrodynamics
\end{itemize}
\end{itemize}
%=================================================================

\section*{Education}
%--------------------------------------------------------------
\begin{itemize}

\item PhD Applied Mathematics, University of Leeds (Oct 2009 -- Sep 2013)
\begin{itemize}
	\item[--] Thesis title: \textit{Shear instabilities in shallow-water
	magnetohydrodynamics}\\
	supervised by David W. Hughes and Stephen D. Griffiths\\
	examined by David G. Dritschel and Chris A. Jones
\end{itemize}
%--------------------------------------------------------------
 \item MMath Mathematics, University of Durham, (First class; Oct 2005 -- Jul 2009)
\begin{itemize}
  	\item[--] MMath dissertation: \textit{Hydrodynamic stability of Newtonian
  	and non-Newtonian fluids}\\
  	supervised by Miguel A. Moyers-Gonzalez
\end{itemize}
%--------------------------------------------------------------
\end{itemize}

%=================================================================

\section*{Research experience}

%--------------------------------------------------------------
\begin{itemize}


\item \textbf{Associate professor} (Jul 2025 -- to date)\\
Department of Ocean Science, Hong Kong University of Science and Technology

\item \textbf{Senior NOC fellow} (visiting position; Sep 2023 -- to date)\\
National Oceanography Centre, UK

\item \textbf{Assistant professor} (cf. lecturer in the UK system; Jul 2019 -- Jun 2025)\\
Department of Ocean Science, Hong Kong University of Science and Technology

\item \textbf{Post-doctoral researcher} (Oct 2017 -- May 2019)\\
AOPP, Department of Physics, University of Oxford\\
-- working primarily with David P. Marshall (PI) and James R. Maddison (Co-I) 

\item \textbf{Post-doctoral researcher} (Sep 2014 -- Sep 2017)\\
School of Mathematics, University of Edinburgh\\
-- working with James R. Maddison (PI) and David P. Marshall (Co-I)

\item \textbf{Post-doctoral researcher} (Oct 2013 -- Sep 2014)\\
Department of Geophysics \& Planetary Sciences, Tel Aviv University\\
-- working with Nili Harnik and Eyal Heifetz

\item \textbf{PhD student}  (Oct 2009 -- Sep 2013)\\
Department of Applied Mathematics, University of Leeds
	
\item \textbf{ISIMA 2010 student research fellow} (Jul -- Aug 2010)\\
University of California, Santa Cruz\\
-- \textit{Geostrophic turbulence under the influence of a magnetic field},
supervised by Patrick H. Diamond
	
\item \textbf{EPSRC vacation studentship} (Jul -- Sep 2008)\\
Department of Mathematical Sciences, University of Durham\\
-- \textit{Hydrodynamic stability of Kolmogorov flows on the sphere},
supervised by Djoko Wirosoetisno
	
%--------------------------------------------------------------
\end{itemize}

%=================================================================
\section*{Publications}
%-----------------------------------------------------------------
\subsection*{Journal Articles}

(${}^\S$ denotes corresponding author(s), bold denotes group members/visitors for the relevant period)

\begin{itemize}

%\item[*] \textbf{J. Mak}$^{\S}$, D. G. Dritschel, N. Harnik, E. Heifetz, \textbf{E. Ong} \& Y. Wang (in prep.)\\
%\textit{Baroclinic spindown in a two-layer quasi-geostrophic system:
%phenomenology and equilibration}

\item[*] R. Torres$^{\S}$, R. Waldman, G. Madec, C. de Lavergne, R. S\'ef\'erian \& \textbf{J. Mak} (to be resubmitted to J. Adv. Model. Earth Syst.)\\
\textit{Energetically constrained mesoscale parameterisations in ocean global circulation models}

\item[*] \textbf{H. S. Lee}$^{\S}$, J. R. Maddison, \textbf{J. Mak}$^{\S}$, D. P. Marshall \& Y. Wang (in revision with Tellus A)\\
\textit{Negative sensitivity of Southern Ocean circumpolar transport to increased wind stress controlled by residual overturning}

\item[27.] \textbf{F. E. Yan}$^{\S}$, H. Frezat, J. Le Sommer, \textbf{J. Mak}$^{\S}$ \& K. Otness (accepted by J. Adv. Model. Earth Syst.)\\
\textit{Adjoint-based online learning of two-layer quasi-geostrophic baroclinic turbulence}

\item[26.] R. Liu, Y. Wang$^{\S}$, X. Zhai, D. Balwada \& \textbf{J. Mak} (2025)\\
\textit{Improved theoretical estimates of the zonal propagation of global nonlinear mesoscale eddies}\\
J. Geophys. Res. Oceans, \textbf{130}(6), e2025JC022518

\item[25.] J. R. Maddison$^{\S}$, D. P. Marshall, \textbf{J. Mak} \& K. Maurer-Song (2025)\\
\textit{A two-dimensional model for eddy saturation and frictional control in the Southern Ocean}\\
J. Adv. Model. Earth Syst., \textbf{17}(4), e2024MS004682

\item[24.] J. Thomy, F. Sanchez, C. Prioux, S. Yau, Y. Xu, \textbf{J. Mak}, R. Sun, G. Piganeau$^{\S}$ \& C. C. M. Yung$^{\S}$ (2024)\\
\textit{Unveiling Prasinovirus diversity and host specificity through targeted enrichment in the South China Sea}\\
ISME Communications, \textbf{4}(1), ycae109

\item[23.] \textbf{J. Mak}$^{\S}$, N. Harnik, E. Heifetz, \textbf{G. Kumar}$^{\S}$ \& \textbf{E. Q. Y. Ong} (2024)\\
\textit{Edge-wave phase shifts versus normal-mode phase tilts in an Eady problem with a sloping boundary}\\
Physical Review Fluids, \textbf{9}(8), 083905

\item[22.] \textbf{X. Ruan}$^{\S}$, D. Couespel, M. L\'evy, \textbf{J. Mak}$^{\S}$ \& Y. Wang (2024)\\
\textit{Combined physical and biogeochemical assessment of mesoscale eddy parameterisations: eddy induced advection at eddy-permitting resolution}\\
Ocean Modelling, \textbf{190}, 102396

\item[21.] \textbf{F. E. Yan}$^{\S}$, \textbf{J. Mak}$^{\S}$ \& Y. Wang (2024)\\
\textit{On the choice of training data for machine learning of geostrophic mesoscale turbulence}\\
Journal of Advances in Modeling Earth Systems, \textbf{16}(2), e2023MS003915

\item[20.] H. Wei, Y. Wang$^{\S}$ \& \textbf{J. Mak} (2024)\\
\textit{Parameterizing eddy buoyancy fluxes across prograde shelf/slope fronts using a slope-aware GEOMETRIC closure}\\
Journal of Physical Oceanography, \textbf{54}(2), 359--377

\item[19.] \textbf{J. Mak}$^{\S}$, J. R. Maddison, D. P. Marshall, \textbf{X. Ruan}, Y. Wang \& L. Yeow (2023)\\
\textit{Scale-awareness in an eddy energy constrained mesoscale eddy parameterization}\\
Journal of Advances in Modeling Earth Systems, \textbf{15}(12), e2023MS003886

\item[18.] R. Torres$^{\S}$, R. Waldman, \textbf{J. Mak}, \& R. S\'ef\'erian (2023)\\
\textit{Global estimate of eddy kinetic energy dissipation from a diagnostic energy balance}\\
Geophysical Research Letters, \textbf{50}(20), e2023GL104688

\item[17.] \textbf{X. Ruan}$^{\S}$, D. Couespel, M. L\'evy, \textbf{J. Mak}$^{\S}$ \& Y. Wang (2023)\\
\textit{Combined physical and biogeochemical assessment of mesoscale eddy parameterisations: eddy induced advection in non-eddying models}\\
Ocean Modelling, \textbf{183}, 102204

\item[16.] H. Wei, Y. Wang$^{\S}$, A. L. Stewart \& \textbf{J. Mak}
(2022)\\
\textit{Scalings for eddy buoyancy fluxes across prograde shelf/slope fronts}\\
Journal of Advances in Modeling Earth Systems, \textbf{14}(12), e2022MS003229

\item[15.] E. Heifetz$^{\S}$, L. R. M. Maas, \textbf{J. Mak} \& I. Pomerantz (2022)\\
\textit{Inertio-gravity Poincare waves and the quantum relativistic Klein-Gordon equation, near-inertio waves and the non-relativistic Schrodinger equation}\\
Physics of Fluids, \textbf{34}, 116608

\item[14.] \textbf{J. Mak}$^{\S}$, A. Avdis, T. David, \textbf{H. S. Lee}, \textbf{Y. Na}, Y. Wang \& \textbf{F. E. Yan} (2022)\\
\textit{On constraining the mesoscale eddy energy dissipation time-scale}\\
Journal of Advances in Modeling Earth Systems, \textbf{14}(11), e2022MS003223

\item[13.] \textbf{J. Mak}$^{\S}$, D. P. Marshall, G. Madec \& J. R. Maddison (2022)\\
\textit{Acute sensitivity of global ocean circulation to eddy energy dissipation time-scale}\\
Geophysical Research Letters, \textbf{49}(8), e2021GL097259

\item[12.] E. Heifetz$^{\S}$, L. R. M. Maas, \textbf{J. Mak} (2021)\\
\textit{Zero absolute vorticity plane Couette flow as an hydrodynamic representation of quantum energy states under perpendicular magnetic fields}\\
Physics of Fluids, \textbf{33} (12), 127120

\item[11.] E. Heifetz, L. R. M. Maas, \textbf{J. Mak}$^{\S}$ \& I. Pomerantz (2021)\\
\textit{On a formal equivalence between electro-magnetic waves in cold unmagnetized plasma and shallow water inertio-gravity waves}\\
Journal of Physics Communications, \textbf{5} (12), 125006

\item[10.] Y. Y. Cheung, S. Cheung$^{\S}$, \textbf{J. Mak}, K. Liu, X. Xia, X. Zhang, Y. Yung \& H. Liu$^{\S}$ (2021)\\
\textit{Distinct interaction effects of warming and anthropogenic input on
diatoms and dinoflagellates in an urbanized estuarine ecosystem}\\
Global Change Biology, \textbf{27} (15), 3463--3473

\item[9.] \textbf{J. Mak}$^{\S}$, J. R. Maddison, D. P. Marshall \& D. R. Munday
(2018)\\
\textit{Implementation of a geometrically informed and energetically constrained
mesoscale eddy parameterization in an ocean circulation model}\\
Journal of Physical Oceanography, \textbf{48}, 2363--2382

\item[8.] \textbf{J. Mak}$^{\S}$, S. D. Griffiths \& D. W. Hughes
(2017)\\
\textit{Vortex disruption by magnetohydrodynamic feedback}\\
Physical Review Fluids, \textbf{2}, 113701

\item[7.] \textbf{J. Mak}$^{\S}$, D. P. Marshall, J. R. Maddison \& S. D. Bachmann
(2017)\\
\textit{Emergent eddy saturation from an energy constrained eddy parameterisation}\\
Ocean Modelling, \textbf{112}, 125--138

\item[6.] S. D. Bachman$^{\S}$, D. P. Marshall, J. R. Maddison and \textbf{J. Mak}
(2017)\\
\textit{Evaluation of a scalar eddy diffusivity based on geometric constraints}\\
Ocean Modelling, \textbf{109}, 44--54

\item[5.] \textbf{J. Mak}$^{\S}$, J. R. Maddison \& D. P. Marshall
(2016)\\
\textit{A new gauge-invariant method for diagnosing eddy diffusivities}\\
Ocean Modelling, \textbf{104}, 252--268

\item[4.] \textbf{J. Mak}$^{\S}$ , S. D. Griffiths \& D. W. Hughes
(2016)\\
\textit{Shear flow instabilities in shallow-water magnetohydrodynamics}\\
Journal of Fluid Mechanics, \textbf{788}, 767--796

\item[3.] E. Heifetz \& \textbf{J. Mak}$^{\S}$ (2015)\\
\textit{Stratified shear flow instabilities in the non-Boussinesq regime}\\
Physics of Fluids, \textbf{27}, 086601, 1--15

\item[2.] E. Heifetz, \textbf{J. Mak}$^{\S}$, J. Nycander \& O. M.
Umurhan (2015)\\
\textit{Interacting vorticity waves as an instability mechanism for
magnetohydrodynamic shear instabilities}\\
Journal of Fluid Mechanics, \textbf{767}, 199--225

\item[1.] M. A. Moyers-Gonzalez$^{\S}$ , T. Burghlea \& \textbf{J. Mak}
(2011)\\
\textit{Linear stability analysis for plane-Poiseuille flow of an
elastoviscoplastic fluid with internal microstructure at large Reynolds
Number}\\
Journal of Non-Newtonian Fluid Mechanics, \textbf{166}, 515-531
\end{itemize}

\subsection*{Reports, proceedings and grey literature}
\begin{itemize}

\item[F.] NEMO Consortium (2022)\\
NEMO Development Strategy 2023-2027 (Version 3)\\
Zenodo (\url{https://doi.org/10.5281/zenodo.7361464})

\item[E.] D. P. Marshall$^{\S}$, J. R. Maddison, \textbf{J. Mak}, S. D. Bachman \& D. R. Munday (2020)\\
\textit{GEOMETRIC: Geometry and energetics of ocean mesoscale eddies and their representation in climate models}\\
CLIVAR exchanges, \textbf{77}, 17-22 (joint special edition on ``Sources and Sinks
of Ocean Mesoscale Eddy energy'')

\item[D.] E. Heifetz \& \textbf{J. Mak}$^{\S}$ (2014)\\
\textit{Magnetohydrodynamic shear instabilities arising from interacting
vorticity waves}\\
Advances in Fluid Mechanics X (Proceedings of AFM2014), 371-381

\item[C.] \textbf{J. Mak}$^{\S}$ (2013)\\
\textit{Shear instabilities in shallow-water magnetohydrodynamics}\\
PhD thesis, Department of Applied Mathematics, University of Leeds

\item[B.] \textbf{J. Mak}$^{\S}$ (2011)\\
\textit{Geostrophic turbulence in the MHD regime}\\
report / proceedings for ISIMA 2010 (see ISIMA website)

\item[A.] \textbf{J. Mak}$^{\S}$ (2009)\\
\textit{Hydrodynamic stability of Newtonian and non-Newtonian fluids}\\
MMath dissertation, Department of Mathematical Sciences, University of Durham
\end{itemize}

%=================================================================
\section*{Grants}

\begin{itemize}

\item[--] \textit{Energetically consistent coupling of a mesoscale eddy and lee wave parameterization in an IPCC-class global ocean circulation model}\\
1st Jul 2025 --- 30th Jun 2028 (PI, Co-I: Casimir de Lavergne, LOCEAN-IPSL)\\
HKD 792,542 (exclusive of overheads), 36 months\\
HKRGC General Research Fund (16303625)

\item[--] \textit{The role of symmetries in fluid and plasma systems}\\
1st Apr 2025 --- 31st Mar 2026 (PI; RIAM host: Yusuke Kosuga and Yohei Onuki)\\
JPY 200,000 (travel grant), 12 months\\
RIAM international joint research (2025S2-CD-5)

\item[--] \textit{Implementing a slope- and scale-aware mesoscale eddy parameterization in global ocean models}\\
1st Sep 2024 --- 31st Aug 2027 (Co-I, PI: Yan Wang, HKUST)\\
HKD 783,478 (exclusive of overheads), 36 months\\
HKRGC General Research Fund (16307324)

\item[--] \textit{Multi-sensor monitoring, geophysical interpretation and prediction of sea level rise in Hong Kong}\\
1st Jun 2024 --- 30th May 2027 (Co-I, PC: Jianli Chen, PolyU)\\
HKD 6,567,108 (exclusive of overheads), 36 months\\
HKRGC Collaborative Research Fund (C5013-23GF)

\item[--] \textit{Inferring ocean eddy energy dissipation timescale from observations using an inverse method}\\
1st Nov 2023 --- 31st Oct 2024 (Co-I, PI: Xiaoming Zhai, UEA)\\
GBP 3,000 (travel grant), 12 months\\
The Royal Society Kan Tong Po International Fellowship (KTP\textbackslash R1\textbackslash 231008)

\item[--] \textit{Numerical modelling of the influence of secondary surface roughness on urban turbulence and ventilation}\\
1st Jul 2021 --- 30th Jun 2024 (Co-I, then PI)\\
HKD 391,015 (exclusive of overheads), 36 months\\
HKRGC General Research Fund (11308021)

\item[--] \textit{Inferring South China Sea abyssal upwelling via a consistent regional state estimate}\\
1st Jul 2021 --- 30th Jun 2024 (PI, Co-I: Matt Mazloff)\\
HKD 598,015 (exclusive of overheads), 36 months\\
HKRGC General Research Fund (16304021)

\item[--] \textit{Parameterization in grey zone ocean general circulation models}\\
1st Nov 2020 -- 31st Oct 2022 (PI)\\
HKD 400,000 (exclusive of overheads)\\
Center for Ocean Research in Hong Kong and Macau

\item[--] \textit{Probing circulation influences on pollution dispersion}\\
1st Jan 2020 -- 31st Dec 2022 (PI)\\
HKD 700,000 (exclusive of overheads)\\ 
Hong Kong Branch Collaborative Research Fund and Operation Fund

\item[--] \textit{Constraining uncertain parameters in IPCC-class global ocean
circulation models using inverse methods}\\
1st Jul 2020 -- 30th Jun 2023 (PI)\\
HKD 705,710 (exclusive of overheads), 36 months \\
HKRGC Early Career Scheme (26300020)

\end{itemize}

%=================================================================

%-----------------------------------------------------------------
\section*{Selected conference/seminar presentations}
\begin{itemize}

%\item[--] \textit{The modified geostrophic Eady problem revisited}\\
%CGAFD Seminar, School of Mathematics, Exeter, Jun 2025

\item[--] \textit{Influences on biogeochemical responses from mesoscale eddy parameterisations}\\
NEMO Hackathon, UK Met Office, Exeter, Jun 2025

\item[--] \textit{Combined physical and biogeochemical assessment of mesoscale eddy parameterisations in ocean models: Eddy-induced advection at eddy-permitting resolutions}\\
JpGU, Chiba, Japan, Jun 2025

\item[--] \textit{Influences on biogeochemical responses by representations of fluid turbulence}\\
ESS Seminar, CUHK, HK, May 2025

\item[--] \textit{Machine learning of geostrophic turbulence}\\
(Invited talk) AAPPS-DDP 2024, Melaka, Nov 2024

\item[--] \textit{Scale-awareness in an energetically constrained eddy parameterisation}\\
Speaker and organiser, Physical Oceanography Day, HKUST, HK, Jun 2024

\item[--] \textit{Hands-on session on Machine Learning techniques}\\
(Invited tutorial 3 hours) TAPGFD, ICTS, Bangalore, May 2024

\item[--] \textit{The geostrophic Eady problem revisited}\\
(Invited lecture 2 hours) TAPGFD, ICTS, Bangalore, May 2024

\end{itemize}

%-----------------------------------------------------------------
\section*{Selected professional activities}
\begin{itemize}

\item[--] Editor: EGU Ocean Sci. (Aug 24 onwards)

\item[--] Reviewer for: J. Fluid Mech.; Phys. Fluids; Phys. Lett. A; J. Phys. Oceanogr.; Phys. Plasmas; Astrophys. J.; J. Geophys. Res: Oceans; Ocean Modell.; J. Adv. Model. Earth Syst.; IPCC AR6 WG1 (second draft); EGU Ocean Sci.; Geophys. Res. Lett.

\item[--] NEMO (Nucleus for European Modelling of the Ocean) working group on eddy parameterisations (22/23 onwards, co-chair from Apr 25); NEMO developer under the NERC group (23/24 onwards)

\item[--] (Dept. of Ocean Science, HKUST) Discovering Ocean Science summer school (22/23, 23/24), co-UG co-ordinator (22/23), Departmental seminar organiser (20/21 to 23/24), International Research Enrichment track co-ordinator (20/21 to date), MSc committee (21/22), UG committee (20/21 to date), Teaching Faculty appointment and promotion committee (22/23 to date), Student Mentoring Task Force (19/20 to 22/23), physical oceanography group meeting oragniser (20/21 to date)

\item[--] Outreach with the Ocean-3C program in Hong Kong (20/21 placement: Sha Tin College; 21/22 placement: Saint Too Cannan college)

\item[--] (School of Mathematics, University of Edinburgh) Reading group member for the Athena Swan Silver application, post-doc wiki administrator, journal club \textit{Waves \& Mean Flows} organiser 

\end{itemize}

%-----------------------------------------------------------------
%\subsection*{Outreach}
%\begin{itemize}

%\item[--] Discovering Ocean Science summer school (22/23, 23/24: Dept. of Ocean
%Science, HKUST) 

%\item[--] Outreach with the Ocean-3C program in Hong Kong (20/21 placement: Sha
%Tin College; 21/22 placement: Saint Too Cannan college)

%\end{itemize}

%=================================================================

%\section*{Awards}
%\begin{itemize}
%	\item[--] ISIMA 2010 student fellowship at UCSC (\$1,000, Jul -- Aug 2010)
%	\item[--] NORDITA winter school: Dynamos, above, below, and in the
%	laboratory (\textsterling600, Jan 2010)
%	\item[--] STFC DTG studentship: tuition fees, stipend and research costs
%	(ST/F006934/1; Oct 2009 -- Mar 2012)
%	\item[--] Further Maths Network student poster prize, \textsterling200 (May
%	2009)
%	\item[--] EPSRC summer vacation studentship, \textsterling2,000 (Jul-Sep
%	2008)
%	\item[--] TDA (part of the Department of Education), student associate
%	scheme, \textsterling600 (Jan 2007)
%\end{itemize}

%=================================================================

\section*{Teaching Activities}

%-----------------------------------------------------------------
\subsection*{Research mentoring}

\begin{tabularx}{\textwidth}{XXX}
  Floriane Oc\'eane SUDRE & PDRA  & Aug 2025 -- now\\
  RUAN Xi                 & PDRA  & Jul 2025 -- now\\
  Virryna WU Yue          & PDRA  & Feb 2024 -- now\\
  Huanhuan WANG           & PDRA  & Sep 2022 -- Feb 2023\\
  Gautam KUMAR            & PDRA  & Mar 2021 -- Sep 2022\\
  \\
\end{tabularx}

\begin{tabularx}{\textwidth}{XXX}
  LEUNG Wai Hang          & MPhil & Sep 2025 -- now\\
  YIN Jiahui              & MPhil & Sep 2022 -- now\\
  Jonathan LEE Ho Ching   & PhD   & Feb 2024 -- now\\
  Dan BARTLEY             & MPhil & Sep 2022 -- Aug 2024\\
  Kayla LEE               & MPhil & Sep 2022 -- Aug 2023\\
  YAN Feier               & PhD   & Feb 2021 -- Mar 2025\\
  NA Yongsu               & PhD   & Sep 2020 -- now\\
  RUAN Xi                 & PhD   & Sep 2020 -- May 2025\\
  LIU Yongqi              & MPhil & Sep 2020 -- Aug 2022\\
  Floriane Oc\'eane SUDRE & PhD   & Sep 2020 -- May 2021\\
  LEE Han Seul            & PhD   & Sep 2019 -- now\\
  \\
\end{tabularx}

\begin{tabularx}{\textwidth}{XXX}
  Dan BARTLEY             & RA    & Feb 2022 -- Jul 2022\\
  Chinmayee MALLICK       & RA    & Apr 2020 -- Mar 2021\\
  Floriane Oc\'eane SUDRE & RA    & Mar 2020 -- Aug 2020\\
  \\
\end{tabularx}

\begin{tabularx}{\textwidth}{XXX}
  DONG Zipei        & MSc               & Sep 2020 -- May 2021\\
  JIANG Jinxiao     & MSc               & Sep 2020 -- May 2021\\
  Ellie ONG         & Visiting scholar  & Oct 2020 -- Jan 2021\\
  RUAN Xi           & MSc               & Feb 2020 -- Jul 2020\\
  \\
\end{tabularx}
  
\begin{tabularx}{\textwidth}{XXX}
  LEUNG Wai Hang          & BSc (Capstone; Ocean Sci.) & Feb 2025 -- May 2025\\
  CHAN Chun Ting          & BSc (FYP; Physics)    & Sep 2024 -- May 2025\\
  CHEN Hin Kwan           & BSc (FYP; Ocean Sci.) & Sep 2024 -- May 2025\\
  HO Chung Yan            & BSc (FYP; Ocean Sci.) & Sep 2024 -- May 2025\\
  Hayden SO               & BSc                   & Sep 2023 -- now\\
  Andersen POON           & BSc (UROP; Maths)     & Jul 2023 -- Sep 2023\\
  Soobeom CHUNG           & BSc (FYP; Physics)    & Sep 2022 -- May 2023\\
  Matthew FONG            & BSc (FYP; Physics)    & Sep 2022 -- May 2023\\
  Tim DU                  & BSc (FYP; Env. Sci.)  & Sep 2021 -- May 2022\\
  Anastasia LEUNG         & BSc (UROP; Physics)   & Jul 2021 -- Sep 2021\\
  Rachel TSOI             & BSc (FYP; Env. Sci.)  & Sep 2020 -- May 2021\\
  Jolie NG                & BSc (FYP; Env. Sci.)  & Sep 2020 -- May 2021\\
  Nick KANG               & BSc (UROP; Physics)   & Jul 2020 -- Sep 2020\\
  Haruki SAEGUSA          & BSc (UROP, UG helper) & Jul 2020 -- Sep 2020\\
                          &     (Mech. Eng.     ) & Jun 2021 -- Jan 2022\\
  Kathryn CHOW            & BSc (Capstone; Env. Sci.)  & Feb 2020 -- Jul 2020\\
  \\
\end{tabularx}

%-----------------------------------------------------------------
\subsection*{Hong Kong University of Science and Technology (2019 to date)}
\begin{itemize}
\item Instructor (I), with rough estimate of student numbers
\begin{itemize}
  \item[25/26:] Data Analysis in Ocean Science (3rd year, I, 40), AI and Machine Learning for Ocean Science (4th year, I, 10)
  \item[24/25:] Descriptive Physical Oceanography (2nd year, I, 40), Data Analysis in Ocean Science (3rd/4th year, I, 8), Physical Oceanography (3rd/4th year, I, 4)
  \item[23/24:] The Earth as a Blue Planet (1st year, Co-I, 70, both semesters), Descriptive Physical Oceanography (2nd year, I, 40), Data Analysis in Ocean Science (3rd/4th year, I, 12), Physical Oceanography (postgraduates, I, 5)
  \item[22/23:] Descriptive Physical Oceanography (2nd year, I, 25), Data Analysis in Ocean Science (3rd/4th year, I, 5), Physical Oceanography (3rd/4th year, I, 5)
  \item[21/22:] Descriptive Physical Oceanography (2nd year, I, 25), Data Analysis in Ocean Science, (3rd/4th year, I, 5)
  \item[20/21:] Postgraduate Seminar (PG, Co-I, 40), Global Climate Change (3rd year, Co-I, 30), Descriptive Physical Oceanography (2nd year, I, 40)
  \item[19/20:] Postgraduate Seminar (PG, Co-I, 40), Global Climate Change (3rd year, Co-I, 30)
\end{itemize}
\end{itemize}

%-----------------------------------------------------------------
\subsection*{University of Edinburgh (2014 to 2017)}
\begin{itemize}
\item Instructor (I), Teaching assistant (T), assignment marker (M), with rough estimate of
student numbers
\begin{itemize}
  \item[16/17:] Mathematics in Action : Mathematics of Climate (honours, I, 30)
	\item[15/16:] Several variable calculus and differential equations
	(pre-honours, TM, 15), Computing and numerics (pre-honours, TM, 60)
	\item[14/15:] Computing and numerics (pre-honours, TM, 60)
\end{itemize}
\end{itemize}

%-----------------------------------------------------------------
\subsection*{University of Leeds (2009 to 2013)}
\begin{itemize}
\item Teaching assistant (T), assignment marker (M), with rough estimate of
student numbers
\begin{itemize}
	\item[12/13:] 1H Mathematics 1 (TM, 10), introduction to linear algebra (TM,
	10, exam marking, 85); 2H calculus of variations (TM, 30)
	\item[11/12:] 2H Fourier series, PDEs and transforms (TM, 30), calculus of
	variations (M, 30), multiple integrals and vector calculus (M, 30)
	\item[10/11:] 1H numbers and vectors (TM, 20), modelling and investigations
	(TM, 60 over the semester); 2H Fourier series, PDEs and transforms (TM, 30),
	mathematics for Geoscience (M, 15), introduction to optimisation (M, 40),
	calculus of variations (M, 30), multiple integrals and vector calculus (M,
	30)
	\item[09/10:] 1H linear algebra, calculus, differential equations and
	mechanics (TM, 7); 2H Introduction to optimisation (M, 40); 3/4H
	Hydrodynamic stability (M, 15).
\end{itemize}
\end{itemize}

%-----------------------------------------------------------------

\subsection*{University of Durham (2008-2009)}
\begin{itemize}
\item Assignment marker
\begin{itemize}
	\item[--] Single maths courses: complex analysis (2H, 25), analysis in many
	variables (2H, 25), algebra and number theory (2H, 25)
\end{itemize}
\end{itemize}

%-----------------------------------------------------------------

\subsection*{Training and development agency for schools (2006-2007)}
\begin{itemize}
	\item Student associate teachers (for Durham and Lincolnshire county
	council, UK)
\begin{itemize}
	\item[--] Government scheme to promote higher education and outreach to
	secondary school students and to provide teaching experience for student
	associates. Acted as academic and pastoral mentor at a school listed as
	`deprived', as well as a successful school.
\end{itemize}
\end{itemize}

%-----------------------------------------------------------------

%\subsection*{Some recent student testimonies}
%\begin{itemize}
%  \item \emph{Really responsible prof who explain and write answer for the exam
%  and assignments . In my experience, not many profs are doing this. I really
%  enjoy the learning environment in the course that students can take their time
%  to finish the assignment and learn something actively outside the course
%  content. Respect! Careful design on the questions in assignments.} -- SFQ from
%  ENVS 3004 Spring 20/21
%  
%  \item \emph{I am Daniel LAM Yin Hoi, one of the students in course ENVS3004. I
%  am expressing my gratitude for your detailed lesson in the first
%  half-semester. Your lecture content is interesting and meaningful which have
%  enhanced my knowledge of atmospheric science. It would be an honor if I have
%  an opportunity to participate in your course in the future again. Thank you
%  one more time for your interesting lecture this semester.} -- Daniel Lam,
%  student from ENVS 3004 Spring 20/21
%  
%  \item \emph{Although the course is really challenging and my marks are low, I
%  still think it is a nice course for me. Indeed I learned a lot, but I think if
%  there is more mathematical stuff will be better, not just theoretical stuff.}
%  -- SFQ from OCES 2003 Spring 20/21

%  \item \emph{Professor Mak tries to use different materials to make the
%  contents look easier for us, which is quite good} -- SFQ from OCES 2003 Spring
%  20/21
%  
%  \item \emph{This course shows that different systems in the atmosphere are
%  correlated and that's really amazing e.g. stratosphere and troposphere are
%  closely related. This course improves my critical thinking and I can learn how
%  to use different perspectives to make judgments. It's nice to learn about some
%  kind of models, proxies, and the pros and cons of using them} -- SFQ from ENVS
%  3004 Spring 19/20
%  
%  \item \emph{I would like to say thank you so much for your great teaching !!!!
%  I am so glad for receiving a good mark and learning a lot about the
%  atmosphere! Thanks a lot for your support ,encouragement and valuable advice
%  on my writing :) I am more interested in Oceanography and climate change now!}
%  -- Viola Lee, student from ENVS 3004 Spring 19/20

%\end{itemize}

%=================================================================
%\section*{Misc.}
%\begin{itemize}
%	\item[--] Fluent in English and Cantonese, basic Mandarin Chinese (can read
%	both traditional and simplified scripts), basic French.
%	\item[--] Experience with Python, FORTRAN 90/95, NEMO, MATLAB, MITgcm, Maple,
%	\LaTeX, UNIX systems
%\end{itemize}
%=================================================================
\section*{Referees}
See separate document for referee contact details.
%\begin{enumerate}
%	\item Prof. David Hughes (principal supervisor)\\
%	Dept. of Applied Mathematics,\\
%	University of Leeds,\\
%	Woodhouse Lane, Leeds,\\
%	UK. LS2 9JT\\
%	(e-mail) \verb|dwh@maths.leeds.ac.uk|\\
%	(tel) +44 1133 435 105\\
%	
%	\item Dr. Stephen Griffiths (co-supervisor)\\
%	Dept. of Applied Mathematics,\\
%	University of Leeds,\\
%	Woodhouse Lane, Leeds,\\
%	UK. LS2 9JT\\
%	(e-mail) \verb|sdg@maths.leeds.ac.uk|\\
%	(tel) +44 1133 435 186\\
%	
%	\item Dr. Andrew Baczkowski (PG Teaching Co-ordinator)\\
%	Dept. of Applied Mathematics,\\
%	University of Leeds,\\
%	Woodhouse Lane, Leeds,\\
%	UK. LS2 9JT\\
%	(e-mail) \verb|a.j.baczkowski@leeds.ac.uk|\\
%	(tel) +44 1133 435 156\\
%
%	\item Dr. Nili Harnik (post-doc supervisor)\\
%	Department of Geophysics \& Planetary Sciences,\\
%	Tel Aviv University,\\
%	Tel Aviv, 69978,\\
%	Israel\\
%	(e-mail) \verb|harnik@tau.ac.il|\\
%	(tel) +972 3 640 6359\\
%
%	\item Prof. Eyal Heifetz (post-doc supervisor)\\
%	Department of Geophysics \& Planetary Sciences,\\
%	Tel Aviv University,\\
%	Tel Aviv, 69978,\\
%	Israel\\
%	(e-mail) \verb|eyalh@post.tau.ac.il|\\
%	(tel) +972 3 640 7048\\
%	
%	\item Prof. David Dritschel (PhD examiner)
%	School of Mathematics and Statistics,\\
%	University of St Andrews,\\
%	North Haugh, St Andrews,\\
%	KY16 9SS,
%	Scotland\\
%	(e-mail) \verb|dgd@mcs.st-and.ac.uk|\\
%	(tel) +44 1334 463 721\\
%\end{enumerate}


\end{document}
